\section{Sección numerada}

\subsection{Subsección}

\subsubsection{Subsubsección}

\paragraph{Párrafo normal} y continúa.

\subparagraph{Subpárrafo} continúa normal.

\section*{Sección no numerada}

\section{Listas}
\label{ssec:listas}

Para hacer una lista normal:

\begin{itemize}
	\item Un elemento
	\item Otro elemento
	\item Uno más
\end{itemize}

\begin{enumerate}
	\item Primer elemento
	\item Segundo elemento
	\item Tercer elemento
\end{enumerate}

\begin{description}
	\item[Un] elemento
	\item[Dos] segundo elemento
\end{description}

\begin{itemize}
	\item Elemento uno
	\begin{enumerate}
		\item uno uno
		\item uno dos
		\begin{enumerate}
			\item con letras
		\end{enumerate}
	\end{enumerate}
	\item Otro elemento
\end{itemize}

\section{Matemáticas}

Una ecuación inline $a = b^2 + c$ otra forma más latexiana \(a = b^2 + c\)

Una ecuación en modo display \[E = mc^2\] en donde la ecuación aparece independiente.

\begin{equation}
	E = mc^2 \label{eq:relativo}
\end{equation}

Como mostrar casos:

\begin{equation}
	f(x) = 
	\begin{cases}
		\frac{x}{2}	& x \geq 1 \\
		1			& x < 1
	\end{cases}
\end{equation}

Modelos matemáticos:

\begin{alignat}{3}
	\text{Min } & \sum_{i = 1}^{m} \sum_{j = 1}^{n} c_{ij}x_{ij} + y_{i} \\
	\text{such that:} \nonumber \\
				& \prod_{i = 1}^{m} \sum_{j = 1}^{n} x_{ij} = n\\ \label{eq:restr2}
				& \sum_{i = 1}^{m} x_{ij} \geq 0 &\forall i \in N\\
				&x_{ij} \in \{0,1\}
\end{alignat}

Detextify (para obtener los comandos en \LaTeX de los símbolos matemáticos): http://detexify.kirelabs.org/classify.html

\section{Imágenes}

\begin{center}
	\includegraphics[width =0.25\textwidth]{./img/blue.png}
\end{center}

\begin{figure}[H]
	\centering
	\includegraphics[width =0.25\textwidth]{./img/green.png}
	\caption{Un cuadro verde.}
\end{figure}

Para crear subfiguras:

\begin{figure}[H]
	\centering
	\begin{subfigure}{0.3\textwidth}
		\includegraphics[width=\textwidth]{./img/blue.png}
		\caption{Azul}
	\end{subfigure}
	\begin{subfigure}{0.3\textwidth}
		\includegraphics[width=\textwidth]{./img/green.png}
		\caption{Verde}
	\end{subfigure}
	\begin{subfigure}{0.3\textwidth}
		\includegraphics[width=\textwidth]{./img/orange.jpg}
		\caption{Naranja}
	\end{subfigure}
	\caption{Una figura con subfiguras.}
	\label{fig:colores}
\end{figure}

\section{Tablas}

Tabla común

\begin{tabular}{|l|c|rr|}
	\hline
	A & B & C & D\\
	\hline
	1 & 3 & 4 & 5\\
	6 & 7 & 8 & 9\\
	\hline
\end{tabular}

Generador de tablas: http://www.tablesgenerator.com/

Tabla con booktabs:

\begin{table}[H]
	\centering
	\begin{tabular}{llr}
		\toprule
		\multicolumn{2}{c}{Item} &            \\ \midrule
		Animal     & Description & Price (\$) \\ \midrule
		Gnat       & per gram    & 13.65      \\
		& each        & 0.01       \\
		Gnu        & stuffed     & 92.50      \\
		Emu        & stuffed     & 33.33      \\
		Armadillo  & frozen      & 8.99       \\ \bottomrule
	\end{tabular}
	\caption{Tabla bonita}
	\label{tab:tabbonita}
\end{table}

\section{Referencias}

Para hacer referencia a Ecuaciones (eq:), Figuras (fig:), Tablas (tab:), Secciones (sec:), etc.

Para hacer referencia a la ecuación \ref{eq:relativo} en la página \pageref{eq:relativo}. O una referencia a una subsección \ref{ssec:listas}.

\section{Bibliografía}

Aquí comenzó todo \citep{turing1937}. Una referencia a otro artículo de \citet{cook1971}.

Una referencia en espanglish \citet{garey1979}.

%%\bibliography{references.bib}

\begin{verbatim}

$ pdflatex main.tex
$ biber main
$ pdflatex main.tex
$ pdflatex main.tex
\end{verbatim}

%\printbibliography

\section{Ayuda}

\begin{verbatim}
$ texdoc paquete
\end{verbatim}

En la página oficial de CTAN: https://ctan.org/